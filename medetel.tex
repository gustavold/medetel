\documentclass[12pt]{article}
\usepackage{sbc-template}
\usepackage{graphicx,url}
\usepackage[brazil]{babel}
\usepackage[utf8]{inputenc}
\usepackage{listings}
\usepackage{color}
\usepackage{scalefnt}
\usepackage{soul}
\sloppy

\title{Borboleta and SaguiSaúde - Open Souce Mobile Telehealth for Public Home Healthcare}

\author{Gustavo Luiz Duarte, Rafael Correia, Pedro Leal, Helves Domingues, \\
		Fabio Kon, Rubens Kon, João Eduardo Ferreira}

\address{
        Departamento de Ciência da Computação, Instituto de Matemática e Estatística\\
    Universidade de São Paulo
    \email{\{gduarte, rafaeljpc, pedro.leal, hhdomingues, kon, rkon, jef\}@ime.usp.br}
}

\begin{document}

\maketitle

\begin{abstract}
   	Healthcare Centers play a major role in the Brazilian public healthcare system
	Healthcare Centers play a major role in the Brazilian public healthcare system
	Healthcare Centers play a major role in the Brazilian public healthcare system
	Healthcare Centers play a major role in the Brazilian public healthcare system
	Healthcare Centers play a major role in the Brazilian public healthcare system
	Healthcare Centers play a major role in the Brazilian public healthcare system
\end{abstract}

\section{Introduction}

% cenário e motivação
Healthcare Centers play a major role in the Brazilian public healthcare system as they are responsible for the primary care in their geographic region. Governmental initiatives such as the Family Health Program have produced outstanding results in the improvement of health indexes by focusing on preventive medicine. In these programs, health professionals and specially-trained community agents visit the homes of patients (mostly in low-income, unfavored neighborhoods) to provide health services. However, at the current stage, these actions are carried out with almost no support from Information Technology. All the data is hand-written in forms that are stored in piles of thousands of pieces of paper that are hardly ever used for any significant health action or study.

% hipotese/objetivos
The Borboleta project conducted by the University of São Paulo, Brazil, aims at developing a mobile Open Source Integrated System for management of health information in the context of public healthcare centers and home healthcare service. The hypothesis we want to verify is that automating data collection and processing can improve significantly the quality of the service provided to the population. To achieve that objective, the system we are developing includes a multimedia electronic health record (EHR), which stores patient personal and health data, including treatment history. The mobile EHR improves the quality of the health service, facilitating access to patient health information and guaranteeing that less data is lost due to hand-written records that are not processed. It will also help to identify the evolution of diseases as the health information database is linked to temporal and geographical information.

\section{System architecture}
% abordagem, proposta
%TODO: nao colocar detalhes de implementação aqui (java lwuit)
The system is composed of two major parts: Borboleta, that runs on smartphones with the Java LWUIT technology and SaguiSaúde that runs on the Healthcare center and is accessed by health professionals via commodity Web browsers.

\subsection{Server}
%% Sagui
The system manages patient personal and socioeconomic data, health appointment scheduling and description, drug prescriptions, lab exams, and health professional information
%TODO: encaixar direito a referencia
\cite{sagui}. All this data is stored in a PostgreSQL database that is managed by health professionals via a user-friendly Web interface implemented using the Ruby on Rails technology.

\subsection{Mobile}
%% Borboleta
The Borboleta module \cite{correia08} aim to be a mobile Eletronic Health Record system, which runs on smartphones
and PDAs, changing the paper forms that was used before. So the health care providers will gain mobility, because a
mobile device is smaller than a bunch of paper forms, and agility, as the system is optimized to not need so much
typing on the inputs.

The module carry a subset of the information stored in the central database. This subset is defined based on the
homes that will be visited in that particular day, then the data is transfered to the mobile system through the
Wi-Fi network at the Healthcare Center. From this step the Borboleta works disconnected from the server module.
At the patient home, the health professional has no network access, for this reason we made the mobile modules
without a network connection, so they can freely see and update the patient data. At the end of the day, health
professionals goes back to the Health Center and synchronize the collected data.

Another aspect that we focus our work is interface of the system. To better fit to the healthcare providers needs
the interface must legible and easy to collect. As long as we developed the system under JavaME technology,
we could use the default interface toolkit, known as LCDUI, but this toolkit does not have all the requirements
to fit our needs, so we look after another interface framework and found the LWUIT, which has a better look and
much more interesting component.

\subsection{Synchronizing}
%Sincronização
A two-way synchronization protocol based on REST (Resource State Transfer) over HTTP is used to exchange information among the mobile and fixed parts of the system.




\section{Results}
% Resultados, Estado atual do sistema
The software is developed using a methodology based on Agile Methods in which preliminary versions of the system are tested by real doctors and nurses monthly and several releases are produced each year. Although SaguiSaúde and Borboleta are not yet in production, some of their modules are already usable and tests are being conducted with a 120,000 people database from the University Healthcare center. The system is available as open source software and can be downloaded with a BSD license from http://ccsl.ime.usp.br/borboleta .

\section{Acknowledgment}
The Borboleta project is supported by Microsoft Research-FAPESP Virtual Institute.

\bibliographystyle{sbcbib}
\bibliography{medetel}

\end{document}
